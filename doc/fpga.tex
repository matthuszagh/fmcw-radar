\section{FPGA}
\label{sec:fpga}

\subsection{XC7A15T-FTG256 Xilinx FPGA}
\label{sec:xc7a15t-ftg256}

There are several different documents for the
\href{https://www.xilinx.com/products/silicon-devices/fpga/artix-7.html?resultsTablePreSelect=documenttype:Data\%20Sheets#documentation}{Artix-7
  FPGA family}, which is used for this radar. There is a
\href{https://www.xilinx.com/support/documentation/user_guides/ug475_7Series_Pkg_Pinout.pdf}{package-pinout document}
that, as its name suggests, contains information about the various package options available and the associated pin
definitions. There is an
\href{https://www.xilinx.com/support/documentation/data_sheets/ds180_7Series_Overview.pdf}{overview sheet} that contains
very general information about the FPGA. There is a
\href{https://www.xilinx.com/support/documentation/user_guides/ug483_7Series_PCB.pdf}{PCB design guide} document that
contains recommendations for soldering the FPGA on a PCB. There is a
\href{https://www.xilinx.com/support/documentation/user_guides/ug480_7Series_XADC.pdf}{XADC document} which describes
the mixed-signal functionality of the FPGA. Finally, there is a
\href{https://www.xilinx.com/support/documentation/user_guides/ug470_7Series_Config.pdf}{configuration guide} that
describes the different ways to configure the FPGA with a bitstream. The FPGA pins follow a logical syntax: they can be
of the form IO\_LXXY\_ZZZ\_\#, IO\_XX\_ZZZ\_\#, or a specific name. If they have a specific name, they have a dedicated
function described starting on page 26 of the package-pinout document. If they have the general form IO\_LXXY\_ZZZ\_\#
or IO\_XX\_ZZZ\_\#, they can be used for several different dedicated functions, or as general purpose IO pins. L
indicates that the pin can be used as a differential pair (differential signaling), where Y=P or N depending on whether
the pin is the positive or negative side of the differential pair. XX gives a unique number identifier that can be used
to associate the two different pins forming a differential pair. ZZZ represents one or more functions that the pin can
be used for in addition to general purpose IO. \# indicates the bank number, which separate the pins into one of several
different regions. The package used in this design is FTG256C, which contains banks 0, 14, 15, 34 and 35. Bank 0
contains the dedicated configuration pins. Each bank has 4 pairs of clock capable inputs for differential or single
ended clocks (there are no global clock pins). The bypass capacitor recommendations are outlined in the PCB design
guide. As with any other bypass capacitor design, place the small caps as close to the pins they connect to as
possible. The larger caps should also be as close as possible, but give priority to the small caps.  The FPGA loads its
configuration from an SPI flash memory device,
\href{http://www.winbond.com/resource-files/w25q32jv\%20revg\%2003272018\%20plus.pdf}{W25Q32JV} The CCLK\_0 pin is
exported from the FPGA to the flash device to drive its operation and coordinates writes and reads to and from the
device. DO is used by the FPGA to perform SPI reads from the flash memory device. It is connected to J14 of the FPGA. DI
is connected to J13 of the FPGA and is used by the FPGA to send data to the flash memory device. CS is active low and is
used by the FPGA to signal data transmission is about to occur. It is connected to L12 on the FPGA. WP (write protect)
and HOLD are both active low pins and are used to prevent the status configuration registers from being written to and
can pause the device when multiple devices share the same SPI signal, respectively They are unused and thus connected to
the 3.3V power supply. The device is powered with 3.3V (it supports a range of 2.7V to 3.6V). Table~\ref{tab:fpga-pins}
contains a list of all FPGA pins and their connections.

\label{tab:fpga-pins}
\begin{tabularx}{\textwidth}{c X>{\raggedright\arraybackslash}X}
        \caption{All FPGA pin connections in alphabetical order.} \\
        \toprule
        \textbf{PIN} & \textbf{DESCRIPTION} \\
        \midrule
        \endhead

        A01 & GND. \\
        A02 & Connected to an external 1x1 pin header that is currently unused by FPGA logic. \\
        A03 & Connected to external 2x4 pin header that is currently unused by FPGA logic. \\
        A04 & Connected to same pin header as A03. Also unused. \\
        A05 & Same. \\
        A06 & One of the voltage supplies for bank 35. It uses 3.3V as with all the other voltage supplies
        to the main banks of the FPGA. \\
        A07 & Connected to same pin header as A03. Unused. \\
        A08 & A different 2x4 pin header. Unused. \\
        A09 & Connected to the same pin header as A08. Unused. \\
        A10 & Floating. \\
        A11 & GND. \\
        A12 & Connected as an output to the FT2232H USB-JTAG converter. Pulling this low allows the
        FT2232H device to output data to the FPGA through the ADBUS pins. Driving it high sets the ADBUS
        pins as inputs. Remember, the ADBUS pins serve as the bidirectional translation between JTAG data
        and USB data. \\
        A13 & SIWU (channel A). This pin is output to the FT2232H device and can be used to optimize USB
        data transfers (more info in the FT2232H datasheet). It is not currently used by the FPGA
        logic. \\
        A14 & Write output pin to the FT2232H device. When this is driven low, the FPGA writes data to the
        FT2232H. When it's driven high, the FPGA can perform a read from the FT2232H. \\
        A15 & Inputs a 60MHz clock signal that originated at the FT2232H. This clock is used to
        synchronize all data transfers between the FPGA and the FT2232H. \\
        A16 & A 3.3V input to bank 15. \\

        \midrule

        B01 & ADC\_SHDN1. One of two ADC\_SHDN outputs that, together with OEA (B02), controls the
        shutdown mode selection of the ADC. It is used by the digital FPGA logic. When both SHDN and OE
        are grounded, the ADC performs normally. When they are both pulled high, the ADC goes into sleep
        mode. There are two other states but they are not used by the FPGA logic. \\
        B02 & ADC\_OE1. See pin B01. This is used for channel A. \\
        B03 & 3.3V input for bank 35. \\
        B04 & Floating. \\
        B05 & Connected to the same 2x4 pin header as A03. Unused. \\
        B06 & Floating. \\
        B07 & Connected to the same 2x4 pin header as A03. Unused. \\
        B08 & GND. \\
        B09 & Connected to the same pin header as A08. Unused. \\
        B10 & Connected to the same pin header as A08. Unused. \\
        B11 & Floating. \\
        B12 & Floating. \\
        B13 & 3.3V input for bank 15. \\
        B14 & RD\#. A read active-low output pin to the FT2232H device. When this is driven low, the FPGA
        reads data from the FT2232H. \\
        B15 & TXE\#. An active-low input from the FT2232H. FT2232H drives this pin low to signal sending
        data to the FPGA (i.e. a read for the FPGA). \\
        B16 & RXF\#. An active-low input from the FT2232H. FT2232H drives this pin low to signal that it
        should read data. To the FPGA, this signals that it should write data. \\

        \midrule

        C01 & Floating. \\
        C02 & OF1. An input from the ADC. It is pulled high when an overflow or underflow occurs at
        channel A. It is currently unused by the FPGA logic. \\
        C03 & Floating. \\
        C04 & Floating. \\
        C05 & GND. \\
        C06 & Floating. \\
        C07 & Floating. \\
        C08 & Floating. \\
        C09 & Floating. \\
        C10 & 3.3V input to bank 15. \\
        C11 & Connected to the same pin header as A08. Unused. \\
        C12 & Connected to the same pin header as A08. Unused. \\
        C13 & Floating. \\
        C14 & Floating. \\
        C15 & GND. \\
        C16 & FT\_SUSPEND. An active-low input to the FPGA. FT223H drives this low when the USB is in
        suspend mode. It is currently unused by the FPGA logic. \\

        \midrule
        D01 & LED. Connects to an LED that is used by the FPGA to signal data processing. \\
        D02 & GND. \\
        D03 & Floating. \\
        D04 & Floating. \\
        D05 & Floating. \\
        D06 & Floating. \\
        D07 & 3.3V input to bank 35. \\
        D08 & Floating. \\
        D09 & Floating. \\
        D10 & Floating. \\
        D11 & Floating. \\
        D12 & GND. \\
        D13 & Floating. \\
        D14 & Floating. \\
        D15 & FT\_D7. Bidirectional pin 7/7 of ADBUS, used to transmit data between the FPGA andFT2232H. \\
        D16 & FT\_D6. Bidirectional pin 6/7 of ADBUS, used to transmit data between the FPGA andFT2232H. \\

        \midrule
        E01 & D10. Input pin 10/11 of the ADC's channel A digital outputs. It is used to transmit data
        from the ADC to FPGA. \\
        E02 & D11. Input pin 11/11 of the ADC's channel A digital outputs. It is used to transmit data
        from the ADC to FPGA. \\
        E03 & Floating. \\
        E04 & 3.3V input to bank 35. \\
        E05 & Floating. \\
        E06 & Floating. \\
        E07 & CFGPVS\_0. A dedicated pin that is part of the configuration logic of the FPGA. It is used
        to specify the voltage level used for all banks. Since we use 3.3V, we connect this pin to the
        same 3.3V voltage level. \\
        E08 & CCLK\_0. An output pin exported from the FPGA to the flash memory device, W25Q32JV, to drive
        its operation and coordinate writes and reads to and from the device. \\
        E09 & GND. \\
        E10 & VCCBRAM. One of the two power supply pins to the FPGA's internal RAM. It requires 1.0V. \\
        E11 & Floating. \\
        E12 & Floating. \\
        E13 & Floating. \\
        E14 & 3.3V power supply to bank 15. \\
        E15 & FT\_D5. Bidirectional pin 5/7 of ADBUS, used to transmit data between the FPGA and
        FT2232H. \\
        E16 & FT\_D4. Bidirectional pin 4/7 of ADBUS, used to transmit data between the FPGA and
        FT2232H. \\

        \midrule

        F01 & VCC0\_35. 3.3V power supply to bank 35. \\
        F02 & D9. Input pin 9/11 of the ADC's channel A digital outputs. It is used to transmit data from
        the ADC to FPGA. \\
        F03 & Floating. \\
        F04 & Floating. \\
        F05 & Floating. \\
        F06 & GND. \\
        F07 & VCCINT. 1.0V power supply for the internal core logic of the FPGA. \\
        F08 & VCCBATT\_0. Can be used for memory backup. We do not use it so we tie it to GND. \\
        F09 & VCCINT. 1.0V power supply for the internal core logic of the FPGA. \\
        F10 & GND. \\
        F11 & VCCBRAM. One of the two power supply pins to the FPGA's internal RAM. It requires 1.0V. \\
        F12 & Floating. \\
        F13 & Floating. \\
        F14 & FT\_D3. Bidirectional pin 3/7 of ADBUS, used to transmit data between the FPGA and
        FT2232H. \\
        F15 & FT\_D0. Bidirectional pin 0/7 of ADBUS, used to transmit data between the FPGA and
        FT2232H. \\
        F16 & GND. \\

        \midrule

        G01 & D10. Input pin 10/11 of the ADC's channel A digital outputs. It is used to transmit data
        from the ADC to FPGA. \\
        G02 & D7. Input pin 7/11 of the ADC's channel A digital outputs. It is used to transmit data from
        the ADC to FPGA. \\
        G03 & GND. \\
        G04 & Floating. \\
        G05 & Floating. \\
        G06 & VCCINT. 1.0V power supply for the internal core logic of the FPGA. \\
        G07 & GNDADC\_0. The reference voltage for the onchip ADC. \\
        G08 & VCCADC\_0. A 1.8V power supply used to power the onchip ADC. \\
        G09 & GND. \\
        G10 & VCCAUX. A 1.8V power supply for auxiliary circuits in the FPGA IC. \\
        G11 & Floating. \\
        G12 & Floating. \\
        G13 & GND. \\
        G14 & Floating. \\
        G15 & FT\_D2. Bidirectional pin 2/7 of ADBUS, used to transmit data between the FPGA and
        FT2232H. \\
        G16 & FT\_D1. Bidirectional pin 1/7 of ADBUS, used to transmit data between the FPGA and
        FT2232H. \\

        \midrule

        H01 & D4. Input pin 4/11 of the ADC's channel A digital outputs. It is used to transmit data from
        the ADC to FPGA. \\

        H02 & D5. Input pin 5/11 of the ADC's channel A digital outputs. It is used to transmit data from
        the ADC to FPGA. \\
        H03 & D6. Input pin 6/11 of the ADC's channel A digital outputs. It is used to transmit data from
        the ADC to FPGA. \\
        H04 & Floating. \\
        H05 & Floating. \\
        H06 & GND. \\
        H07 & VREFN\_0. A dedicated pin that is used as a 1.25V reference GND voltage. Tied to GND. \\
        H08 & VP\_0. A dedicated pin that is used as the XADC differential analog input (positive side). It
        is left floating and is unused. \\
        H09 & VCCINT. 1.0V power supply for the internal core logic of the FPGA. \\
        H10 & DONE\_0. A bidirectional dedicated pin. It is pulled high when configuration is done. It
        is connected to a pull-up resistor and an external connector, presumably for debugging purposes. \\
        H11 & Floating. \\
        H12 & Floating. \\
        H13 & Floating. \\
        H14 & Floating. \\
        H15 & VCC0\_15. 3.3V power supply for bank 15. \\
        H16 & CARD\_DETECT. Connects to the SD card. Currently unused by FPGA logic. \\

        \midrule

        J01 & Floating. \\
        J02 & VCC0\_35. 3.3V power supply for bank 35. \\
        J03 & D3. Input pin 3/11 of the ADC's channel A digital outputs. It is used to transmit data from
        the ADC to FPGA. \\
        J04 & MIX\_ENBL. An enable pin that is output to the ADL5802 mixer. It is pulled low to enable
        the mixer and pulled high to disable it. \\
        J05 & Floating. \\
        J06 & VCCINT. 1.0V power supply for the internal core logic of the FPGA. \\
        J07 & VN\_0. A dedicated pin that is used as the XADC differential analog input (negative
        side). It is tied to GND and is unused. \\
        J08 & VREFP\_0. A dedicated pin that is used as a 1.25V reference input. It is tied to GND and
        unused. \\
        J09 & GND. \\
        J10 & VCCAUX. A 1.8V power supply for auxiliary circuits in the FPGA IC. \\
        J11 & GND. \\
        J12 & VCC0\_15. 3.3V power supply for bank 15. \\
        J13 & SPI\_MOSI. Used by the FPGA to send data to the flash memory device. \\
        J14 & SPI\_DIN. Used by the FPGA to read data from the flash memory device. \\
        J15 & Floating. \\
        J16 & Floating. \\

        \midrule

        K01 & D2. Input pin 2/11 of the ADC's channel A digital outputs. It is used to transmit data from
        the ADC to FPGA. \\
        K02 & D1. Input pin 1/11 of the ADC's channel A digital outputs. It is used to transmit data from
        the ADC to FPGA. \\
        K03 & ADF\_MUXOUT. Input from the frequency synthesizer. It indicates that a sweep is done. \\
        K04 & GND. \\
        K05 & Floating. \\
        K06 & GND. \\
        K07 & DXN\_0. The cathode of two temperature-monitoring diode pins. It is not used and is
        therefore tied to GND. \\
        K08 & DXP\_0. The anode of two temperature-monitoring diode pins. It is not used and is therefore
        tied to GND. \\
        K09 & VCCINT. 1.0V power supply for the internal core logic of the FPGA. \\
        K10 & INIT\_B. Indicates initialization of configuration memory. It is pulled high and is unused. \\
        K11 & VCCAUX. A 1.8V power supply for auxiliary circuits in the FPGA IC. \\
        K12 & Floating. \\
        K13 & Floating. \\
        K14 & GND. \\
        K15 & Floating. \\
        K16 & Floating. \\

        \midrule

        L01 & GND. \\
        L02 & D0. Input pin 0/11 of the ADC's channel A digital outputs. It is used to transmit data from
        the ADC to FPGA. \\
        L03 & Floating. \\
        L04 & Floating. \\
        L05 & Floating. \\
        L06 & VCC0\_0. 3.3V power supply for bank 0 (i.e. the bank for dedicated configuration pins). \\
        L07 & TCK. An input pin originating at the output of FT2232H and is used to transmit the
        JTAG clock. It is not used by the FPGA logic. \\
        L08 & VCCINT. 1.0V power supply for the internal core logic of the FPGA. \\
        L09 & PROGRAM\_B. Connected to a pushbutton switch and can be used to perform an asynchronous
        reset of the configuration logic. \\
        L10 & VCCAUX. A 1.8V power supply for auxiliary circuits in the FPGA IC. \\
        L11 & GND. \\
        L12 & SPI\_CS. An output pin that can be brought low to indicate that a transmission will take
        place between the FPGA and flash storage device. It is currently left in the high-impedance
        state in FPGA logic, seemingly indicating that the flash storage device is not yet used. \\
        L13 & Floating. \\
        L14 & Floating. \\
        L15 & PUDC\_B. Pulled low, which configures all I/O pins to enable their internal pull-up
        resistors. \\
        L16 & VCC0\_14. 3.3V power supply to bank 14. \\

        \midrule

        M01 & OF2. An input from the ADC. It is pulled high when an overflow or underflow occurs at
        channel
        B. It is currently unused by the FPGA logic. \\
        M02 & ADF\_DATA. A serial data output pin the frequency synthesizer. \\
        M03 & VCC0\_34. 3.3V power supply to bank 34. \\
        M04 & Floating. \\
        M05 & Floating. \\
        M06 & Floating. \\
        M07 & TMS. Output pin that is fed into the FT223H as a mode select pin. It is unused by the
        current FPGA logic. \\
        M08 & GND. \\
        M09 & M0. Along with M1 and M2, this specifies the configuration mode of the FPGA. M[2:0] =
        001 which indicates the FPGA acts as a master in an SPI interface. Because a DNP resistor is
        placed between M0 and 3.3V, it is not actually connected to the power supply. However, since all
        pull-up resistors should be enabled by pulling PUDC\_B low, it should register a logic 1. \\
        M10 & M1. See M09. \\
        M11 & M2. See M09. \\
        M12 & Floating. \\
        M13 & VCC0\_14. 3.3V power supply to bank 14. \\
        M14 & Floating. \\
        M15 & Floating. \\
        M16 & SD\_DAT1. A data line to the SD card reader. Not currently used by the FPGA logic. \\

        \midrule

        N01 & ADF\_LE. An output that connects to the ADF4158 frequency synthesizer. When it is
        pulled high, data stored in the ADF4158 shift registers is loaded into one of the 8 latches. \\
        N02 & ADC\_SHDN2. See B01. \\
        N03 & Floating. \\
        N04 & Floating. \\
        N05 & GND. \\
        N06 & Floating. \\
        N07 & TDI. JTAG data input from FT2232H. It is not currently used by the FPGA logic. \\
        N08 & TDO. JTAG data output to FT2232H. It is not currently used by the FPGA logic. \\
        N09 & Floating. \\
        N10 & VCC0\_14. 3.3V power supply for bank 14. \\
        N11 & CLK\_REF. An input pin that takes the main 40MHz reference clock used by the FPGA. It is
        one of the outputs of the clock fanout buffer. \\
        N12 & Floating. \\
        N13 & Floating. \\
        N14 & SD\_CLK. A clock that synchronizes activity with the SD card reader. It is not currently used
        by the FPGA logic. \\
        N15 & GND. \\
        N16 & SD\_DAT0. A data line to the SD card reader. Not currently used by the FPGA logic. \\

        \midrule

        P01 & ADC\_OE2. See pin B01. This is used for channel B. \\
        P02 & GND. \\
        P03 & ADF\_TXDATA. Output pin that transmits data to be used by the ADF4158 frequency
        synthesizer for FSK or PSK transmission. This is unused by the FPGA logic. \\
        P04 & Floating. \\
        P05 & Floating. \\
        P06 & Floating. \\
        P07 & VCC0\_14. 3.3V power supply for bank 14. \\
        P08 & Floating. \\
        P09 & Floating. \\
        P10 & Floating. \\
        P11 & Floating. \\
        P12 & GND. \\
        P13 & Floating. \\
        P14 & Floating. \\
        P15 & Floating. \\
        P16 & SD\_CMD. A pin used to communicate with the SD card reader. Currently unused by the FPGA
        logic. \\

        \midrule

        R01 & ADF\_CLK. An output clock used to synchronize the operation of the ADF4158 frequency
        synthesizer. \\
        R02 & Floating. \\
        R03 & ADF\_CE. An output pin to the ADF4158. When this pin is driven low, it powers down
        the frequency synthesizer. \\
        R04 & VCC0\_34. 3.3V power supply for bank 34. \\
        R05 & Floating. \\
        R06 & Floating. \\
        R07 & Floating. \\
        R08 & Floating. \\
        R09 & GND. \\
        R10 & Floating. \\
        R11 & Floating. \\
        R12 & Floating. \\
        R13 & Floating. \\
        R14 & VCC0\_14. 3.3V power supply for bank 14. \\
        R15 & SD\_DAT3. A data line to the SD card reader. Not currently used by the FPGA logic. \\
        R16 & SD\_DAT2. A data line to the SD card reader. Not currently used by the FPGA logic. \\

        \midrule

        T01 & VCC0\_34. 3.3V power supply for bank 34. \\
        T02 & PA\_OFF. Output pin that connects to the base of a transistor and can be used to
        enable (high) or disable (low) the operation of the power amplifier, SE2567L. \\
        T03 & Floating. \\
        T04 & ADF\_DONE. Input pin that that is unused by the FPGA logic. \\
        T05 & Floating. \\
        T06 & GND. \\
        T07 & Floating. \\
        T08 & Floating. \\
        T09 & Floating. \\
        T10 & Floating. \\
        T11 & VCC0\_14. 3.3V power supply for bank 14. \\
        T12 & Floating. \\
        T13 & Floating. \\
        T14 & Floating. \\
        T15 & Floating. \\
        T16 & GND. \\

        \bottomrule
\end{tabularx}

PROGRAM\_B\_0 is connected to a pushbutton switch and can be used to perform an asynchronous reset
to the configuration logic. TDI, TDO, TMS, and TCK are used for the JTAG clock, data input, data
output, and mode select, respectively. They are connected to the FT2232H IC that translates between
USB signals and JTAG signals. They are used to configure the FPGA (which is also stored in the flash
memory). M[2:0] specify the configuration mode for the FPGA. In our configuration, M0 is pulled high
while the other 2 are pulled low, indicating a Master SPI interface. CFGBVS\_0 is used to specify
the voltage level used for all banks. Since we use 3.3V, we connect this pin to the same 3.3V. The
PUDC\_B pin is pulled low, which configures all I/O pins to enable their internal
pull-upresistors. \textbf{\{STARTINCOMPLETE\}} The FPGA seems to have a built-in ADC. However, it
doesn't appear to be used \textbf{\{END INCOMPLETE\}}. There is an SD card reader connected up to
the FPGA, however, it is unused by the current FPGA code. VCCAUX is used for auxiliary circuits and
must be 1.8V. VCCADC\_0 is also 1.8V and is used to power the onchip ADC. VCCINT powers the internal
core logic of the FPGA and must be 1.0V. VCCBRAM is the power supply for the FPGA internal RAM,
which requires 1.0V.