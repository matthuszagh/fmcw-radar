\section{Top}
\label{sec:top}
\textit{\hyperlink{schematic.1}{schematic}}

\subsection{Overview}
\label{sec:top-overview}

The top-level schematic has two purposes: it instantiates the various submodules in the design and
generates a $40 \si{MHz}$ clock signal that is distributed to the
\hyperref[sec:xc7a15t-ftg256]{FPGA}, \hyperref[sec:ltc2292]{ADC} and
\hyperref[sec:adf4158]{frequency synthesizer}.

The \hyperref[sec:kt2520k]{crystal oscillator} has its own $1.8 \si{V}$ power supply and the
\hyperref[sec:nc7svu04]{inverter} and \hyperref[sec:nb3n551]{clock buffer} share their own
$3.3 \si{V}$ power supply. The LDO used is a very low noise \hyperref[sec:lp5907]{LP5907}. However,
the PSRR of the LDO at the switching frequency of the \hyperref[sec:tps5420d]{buck converter that
  feeds it} is greatly diminished from it's lower frequency value ($80 \si{dB}$ at $1 \si{kHz}$ and
only about $40 \si{dB}$ at $500 \si{kHz}$). So, ferrite bead pi filters are used at the LDO outputs
to provide further noise suppression for higher frequencies. They also prevent noise generated by
the clocks from reentering the power supply and affecting other devices.

\subsection{KT2520K 40MHz TXCO}
\label{sec:kt2520k}

\subsubsection{Description}
\label{sec:kt2520k-description}

The KT2520K crystal oscillator outputs a $40 \si{MHz}$ clipped sine wave. It requires a DC-blocking
capacitor $\geq 1 \si{nF}$ at the output since one is not included internally. I've used $100
\si{nF}$ which is recommended by the \hyperref[sec:ltc2292]{ADC} datasheet.

\subsection{NC7SVU04 Inverter}
\label{sec:nc7svu04}

\subsubsection{Description}
\label{sec:nc7svu04-description}

The NC7SVU04 is a high-speed logic inverter used to convert its clipped sine-wave input into a
square wave output. It has a typical propagation delay of $1.5 \si{ns}$.

\subsubsection{Component Selection}
\label{sec:nc7svu04-component-selection}

The inverter configuration is specified by the \hyperref[sec:ltc2292]{ADC} datasheet. The voltage
divider sets the DC bias point at $V_{\text{CC}}/2$, which in turn sets the duty cycle to 50\%. The
coupling capacitor is required because the crystal oscillator has low operational voltage and it
allows the divider to set the DC bias point.

\subsubsection{PCB Layout}
\label{sec:nc7svu04-pcb}

The oscillator, inverter and buffer should be placed adjacent to the ADC, which is very susceptible
to noise from clock jitter.

\subsection{NB3N551 1:4 Clock Buffer}
\label{sec:nb3n551}

\subsubsection{Description}
\label{sec:nb3n551-description}

The NB3N551 clock buffer duplicates the square wave input clock signal into 3 identical clock
outputs that are fed to the FPGA, ADC and frequency synthesizer.

\subsubsection{Component Selection}
\label{sec:nb3n551-component-selection}

A $33 \si{\Omega}$ source terminating resistor is used for the \hyperref[sec:xc7a15t-ftg256]{FPGA}
and \hyperref[sec:adf4158]{frequency synthesizer} clock lines because the traces leading to their
clock pins are longer than 1 inch. This prevents noise from the FPGA and frequency synthesizer from
affecting the clock fan-out. The source terminating resistor is not used for the ADC line because it
is assumed that the fan-out buffer is immediately adjacent to the ADC\@. However, if the clock is
placed farther away, the $33 \si{\Omega}$ must be used on this line as well.

%%% Local Variables:
%%% mode: latex
%%% TeX-master: "fmcw-radar"
%%% End:
